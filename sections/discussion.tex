
\section{Discussion}
\label{sec:discussion}

The goal of this project was to develop a program for the analysis of night sky images
and use it to estimate the source-count distribution of stars. The program consists of
the calibration of the input image, the identification of stars, matching them to a
catalog and estimating their magnitudes.

Judging by the results, the calibration of the image was successful. The faulty pixels
were corrected, the vignetting effect was removed, and the skyglow was reduced. I think
the biggest improvement would be to verify that the calibration works reliably under
different conditions. How accurate is the method for different cameras, lenses, ISO
settings and exposure times? How well does the method handle diffraction spike patterns?
What if objects obstruct the view or the image is taken in the presence of the moon? I was
not able to test these questions, but I am confident that my approach performs well for
wide-field images with no obstructing objects.

The identification of stars was also successful. The method was able to find the majority
of the stars that are visible in the image. The main problem was the distinction between
stars and noise. I manually found stars that were not recognized, but they were faint and
hard to distinguish from the background. Also, if the image is very blurry or out of
focus, the algorithm might not work well. It would be interesting to combine the flood
fill method with the template matching approach for small objects. This could help to
identify stars that are too faint for the current method. The subsequent steps were very
reliant on the accuracy of solving for the camera orientation. It would be possible to
improve the robustness of the solver by taking the brightness of the stars into account.
Also, the solver could be combined with the catalog matching step to make the intermediate
step unnecessary. When estimating the brightness using the pixel sum, a big reason for
inaccuracy was that the sensor was saturated for many of the bright stars. This introduced
a different slope in the data, which the exponential function could not capture. It could
be modeled by a more complex function, but it would need to avoid overfitting the dimmer
stars. The magnitude estimation does not work for stars with a magnitude lower than 4.0 or
higher than 8.0, due to the lack of training data. The source-count distribution was
estimated by fitting a power-law to the cumulative distribution of the magnitudes. The
slope of the distribution was close to the expected value of $0.6$, but the distribution
was inflated at the bright end. I suspect this was due to the low accuracy of the
predictor. However, the shape of the distribution does fit the expected form.

In conclusion, the program is a good starting point for the analysis of night sky images.
It is able to calibrate images and give a solid estimate of how bright the stars are. The
code is modular and can be used as a library or as a basis for further development.
